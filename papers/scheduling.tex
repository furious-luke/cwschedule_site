\title{A Very Simple \LaTeXe{} Template}
\author{
  Luke Hodkinson \\
  Center for Astrophysics and Supercomputing \\
  Swinburne University of Technology \\
  Melbourne, Hawthorn 32000, \underline{Australia}
}
\date{\today}

\documentclass[12pt]{article}
\usepackage{amsmath}
\usepackage{amsfonts}

\newcommand{\deriv}[2]{\ensuremath{\frac{\mathrm{d}#1}{\mathrm{d}#2}}}
\newcommand{\sderiv}[2]{\ensuremath{\frac{\mathrm{d}^2#1}{\mathrm{d}#2^2}}}

\begin{document}
\maketitle

\begin{abstract}
  Fluid flow optimisation.
\end{abstract}

\section{Introduction}
Difficulting in solving scheduling problem.

\paragraph{Outline}
The remainder of this article is organized as follows.
Section~\ref{previous work} gives account of previous work.
Our new and exciting results are described in Section~\ref{results}.
Finally, Section~\ref{conclusions} gives the conclusions.

\section{Previous work}\label{previous work}
A much longer \LaTeXe{} example was written by Gil~\cite{Gil:02}.

\section{Maths}
For each task, $T_i$, represent its schedule as a density function,
  \[ d_i = d_i\left( t \right) \,. \]
$d_i$ maps onto $\left[0, 1\right]$ and $d_i \in \mathbb{R}$. Also
$t \in \mathbb{R}$ and is in $\left[0, 1\right]$. Also define density
flow as
\begin{eqnarray*}
  u_i & = & u_i\left( t \right) \\
      & = & \deriv{d_i}{t} \,.
\end{eqnarray*}

Require
  \[ \sum_i d_i\left( t_j \right) = 1, \quad \forall j \]
and
  \[ \int_{-\infty}^{\infty} d_i \, \mathrm{d}t = w_i, \quad \forall i . \]

Define a pressure variable used to drive flow, $P = P\left( t \right)$.
This will be the method used to incorporate a kind of fitness function.
  \[ P = P\left( t \right) \]
  \[ f_P = \deriv{P}{t} \]
  \[ \sum_i \mu_i\deriv{d_i}{\tau} = -P \]
  \[ \sum_i \deriv{}{t}\left(\mu_i\deriv{d_i}{t}\right) = -\deriv{P}{t} \]
  \[ \sum_i \mu_i\sderiv{d_i}{t} = -\deriv{P}{t} \]
  \[ \sderiv{}{t}\sum_i\mu_id_i = -\deriv{P}{t} \]
Implies the continuity equations would be
  \[ \deriv{u_i}{t} = \deriv{d_i}{t} \quad \forall i . \]
This shows that divergence or convergence in the density flow must corelate
to a respective decrease or increase in the density.

Alternate description:
  \[ P = P\left( d_i \right) \]
  \[ P_j = \deriv{P}{d_j} \]
  \[ \sum_i \deriv{d_i}{t} = -P \]
  \[ \sum_i \deriv{}{d_j}\left( \deriv{d_i}{t} \right) = -P_j \]

From Stokes flow:

\section{Results}\label{results}
In this section we describe the results.

\section{Conclusions}\label{conclusions}
We worked hard, and achieved very little.

\bibliographystyle{abbrv}
\bibliography{main}

\end{document}
